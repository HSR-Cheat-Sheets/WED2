%! Author = mariuszindel
%! Date = 24.01.21

\section{Internationalization}

\subsection{Definition}
\textbf{Internationalization:} Programmierung, so dass Lokalisierung möglich ist \\
\textbf{Lokal-/Globalisierung:} Sprachliche und andereweitige Anpassung an Notwendigkeit einer Sprachregion.\\
\textbf{Translation:} Übersetzung von Texten oder Textfragmenten\\
\textbf{Locale:} String der eine Sprachregion bestimmt $\rightarrow$ de-CH, de-AT, ...
\subsubsection{Herausforderung}
\begin{itemize}
    \item  Andere Alphabete und Zahlen
    \item  Anpassung Artikel + Adjektive, Pluralbildung
    \item  Andere Farbsymbolik
    \item  Nicht das Konzept von Vorname / Nachname
    \item  Andere Höflichkeitsformen (Sie, Titel, ...)
\end{itemize}
\textbf{Schwierigkeit:} Westeuropa < Osteuropa < Kyrillische < Indische < Arabisch/Hebräisch < Asiatische


\subsection{Tools}
\subsubsection{Intl.Collator}
Sprachsensitiver Stringvergleich
\begin{lstlisting}
const compareGerman = new Intl.Collator('de').compare('ae', 'z');
\end{lstlisting}

\subsubsection{Intl.DateTimeFormat}
Datum und Zeiten sprachsensitiv formatieren
\begin{lstlisting}
const date = new Date(Date.UTC(2019, 11, 16, 15, 30, 0));
console.log(new Intl.DateTimeFormat('en-US').format(date));
\end{lstlisting}

\subsubsection{Intl.ListFormat}
Aufzählungen (und/oder verknüpft) sprachsensitiv formatieren
\begin{lstlisting}
const formatterEn = new Intl.ListFormat('en', {
style: 'long', type: 'conjunction'});
console.log(formatterEn.format(colorsEn));
\end{lstlisting}

\subsubsection{Intl.NumberFormat}
Zahlen sprachsensitiv formatieren
\begin{lstlisting}
const  number = 123456.789;
console.log(new Intl.NumberFormat('de-CH').format(number));
// 123'456.789
\end{lstlisting}

\subsubsection{Intl.PluralRules}
Mit Pluralregeln pluralsensitiv interpolieren
\begin{lstlisting}
function makeOrdinal(n, locale = 'en') {
const pr = new Intl.PluralRules(locale, {
type: 'ordinal' });
const messages = {
en: {
one: 'st', two: 'nd', few: 'rd', other: 'th'},
de: {
one: 'te', two: 'te', few: 'te', other: 'te' }};
return `${n}${messages[locale][pr.select(n)]}`;}
\end{lstlisting}

\subsubsection{Intl.RelativeTimeFormat}
Relative Zeitangaben sprachsensitiv formatieren
\begin{lstlisting}
const rtf1 = new Intl.RelativeTimeFormat("en", {
localeMatcher: "best fit", // other values: "lookup"
numeric: "always", // other values: "auto"
style: "long", // other values: "short" or "narrow" });
console.log(rtf1.format(-1, 'day')); //expected output: "1 day ago"
\end{lstlisting}



\vfill


\columnbreak