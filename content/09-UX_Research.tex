%! Author = mariuszindel
%! Date = 24.01.21

\section{UX Research, Information Architecture}

\subsubsection{Shackel Diagram}
\textbf{User} Wer ist mein Benutzer? (Wissensstand)\\
\textbf{Task} Was sind die Ziele? (Rechnung bezahlen)\\
\textbf{Tool} Was für Tools / Strategien kennen sie?\\
\textbf{Context} In Welchem Zusammenhang benutzt?

\subsubsection{Nielsens „1st Rule of Usability“}
% die Bedeutung von Nielsens „1st Rule of Usability“ („don‘t Listen to Users“) erklären.
Um die beste UX zu gestalten, achten Sie darauf, was Benutzer tun, nicht was sie sagen. Selbstaussagen sind unzuverlässig, ebenso wie Benutzerspekulationen. Benutzer wissen nicht, was sie wollen.

\subsubsection{Fokus Gruppen}
% Sie können zur Nützlichkeit von Fokus Gruppen beim User Research Auskunft geben.
Moderierte Gruppendiskussion mit tatsächlichen oder potenziellen Nutzern zur Erhebung von Anforderungen oder Nutzerfeedback


\subsubsection{Problem-Space vs. Solution-Space}
% den Unterschied von Problem-Space und Solution-Space erklären und warum dieser Unterschied bedeutet, dass Nutzer keine gute Quelle von Design-Vorschlägen sein sollten.
\textbf{Problem-Space:} Der Benutzer ist der Problem-Experte (Bestellung, Bezahlung, Versand)
\textbf{Solution-Space:} Der Designer ist der Lösungs-Experte (Form, Excel, UI-Controls)


\subsubsection{Szenarios}
% den Wert und die Elemente von Szenarios erklären.
\textbf{Storyboard:} Persona, Problem Beschreibung mit Ziel, Trigger, Schritte, Lösung oder Fehler\\
\textbf{Wireframes:} Skizze eines einzelnen Bildschirms\\
\textbf{Screen-Flows:} Sequenz von Bildschirmen

% \subsubsection{Nutzerforschung}
% erklären warum bei Nutzerforschung neben der Toolverwendung auch Eigenschaften, Nutzer- Aufgaben und der Kontext dokumentiert werden sollten.

\subsubsection{Bewertung: Kofferraum Test (nach Krug)}
% Techniken zur Bewertung und Optimierung der Navigationsunterstützung (Fragebogen, Card- Sort, Tree-Testing) in Web-Seites erklären und sinnvoll einsetzen.
• Welche Website?
• welche Seite (Seitenname)?
• Hauptsektionen?
• Optionen auf Level?
• Wo im Übersichtsplan?
• Wie Suche starten?

\subsubsection{Concept-Model vs. Site-Map}
\textbf{Concept-Model} Wichtige Konzepte und deren Relationen (z.B. UML mit Student, Studiengang)\\
\textbf{Site-Map} = Baum/Netz der Seiten


\subsubsection{ISO 9241-11 und Quesenbery}
\textbf{Effektivität:} Benutzer können Ziele erreichen (Messen: \% Korrekt)
\textbf{Effizienz:} Benutzer können Ziele mit angemessenem Aufwand erreichen (Messen: Zeit)
\textbf{Zufriedenheit:} Benutzer sind positiv gegenüber dem System eingestellt (Messen: Fragebogen; Fehlertoleranz auch \% Korrekt)
