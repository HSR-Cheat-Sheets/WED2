%! Author = mariuszindel
%! Date = 24.01.21

\section{UX Research, Information Architecture}


\subsubsection{Nielsens „1st Rule of Usability“}
% die Bedeutung von Nielsens „1st Rule of Usability“ („don‘t Listen to Users“) erklären.
Um die beste UX zu gestalten, achten Sie darauf, was Benutzer tun, nicht was sie sagen. Selbstaussagen sind unzuverlässig, ebenso wie Benutzerspekulationen über zukünftiges Verhalten. Benutzer wissen nicht, was sie wollen.

\subsubsection{Fokus Gruppen}
% Sie können zur Nützlichkeit von Fokus Gruppen beim User Research Auskunft geben.
Nutzer auf das lenken, was er sehen soll $\rightarrow$ Features: entfernen, verbergen, gruppieren oder nur im ausnahmefall anzeigen


\subsubsection{Problem-Space vs. Solution-Space}
% den Unterschied von Problem-Space und Solution-Space erklären und warum dieser Unterschied bedeutet, dass Nutzer keine gute Quelle von Design-Vorschlägen sein sollten.
\textbf{Problem-Space:} Der Benutzer ist der Problem-Experte (Bestellung, Bezahlung, Versand)
\textbf{Solution-Space:} Der Designer ist der Lösungs-Experte (Form, Excel, UI-Controls)


\subsubsection{Szenarios}
% den Wert und die Elemente von Szenarios erklären.
\textbf{Storyboard:} Persona, Problem Beschreibung mit Ziel, Trigger, Schritte, Lösung oder Fehler\\
\textbf{Wireframes:} Skizze eines einzelnen Bildschirms\\
\textbf{Screen-Flows:} Sequenz von Bildschirmen

% \subsubsection{Nutzerforschung}
% erklären warum bei Nutzerforschung neben der Toolverwendung auch Eigenschaften, Nutzer- Aufgaben und der Kontext dokumentiert werden sollten.

\subsubsection{Bewertung und Optimierung}
% Techniken zur Bewertung und Optimierung der Navigationsunterstützung (Fragebogen, Card- Sort, Tree-Testing) in Web-Seites erklären und sinnvoll einsetzen.
\textbf{Optimierung:} Wo bin ich im Moment, Wo kann ich hin, Was ist passiert

