%! Author = mariuszindel
%! Date = 24.01.21

\section{Node JS}

\subsubsection{Event-driven, Non-blocking}
%kennen den Unterschied zwischen klassischen und event-driven Web Services.
Problem single Thread soll gelösst werden.\\
Nur 1 ''Arbeiter'' $\rightarrow$ gibt Anfrage weiter an z.B. FS und soll antwort senden, sobald fertig. Arbeiter nun frei für weitere Anfragen.

\subsubsection{Request \& Response}
\textbf{Request:} GET, PUT, POST, \dots\\
\textbf{Reponse:} writeHead, setHeader, statusCode, statusMessage, write, end
\begin{lstlisting}
response.writeHead(200, {'Content-Length': body.length, 'Content-Type': 'text/plain'});
response.setHeader("Content-Type", "text/html");
response.statusCode = 404; response.statusMessage = ''Not found'';
response.write("Data");
response.end("Data");
\end{lstlisting}


\subsubsection{Module}
Node verwendet für die Module Verwaltung npm
\begin{lstlisting}
export router; // Variable
import router from "./file.js";
export {function, otherFunction};
export const noteService = new NoteService();
import {noteService} from "./noteServices.js";
import express from "express"; // ES6
\end{lstlisting}
\subsubsection{package.json}
\begin{itemize}
    \item Beinhaltet die Informationen zum Projekt
    \item Wird benötigt um es zu publishen
    \item Wird benötigt um Module zu installieren
\end{itemize}




%verstehen den Unterschied zwischem Server und Client JavaScript Code.
%können die Konzepte von Node.js anwenden.
%können mit dem Online API von Node.js umgehen.
%können mit Node.js einen rudimentären Server implementieren.

