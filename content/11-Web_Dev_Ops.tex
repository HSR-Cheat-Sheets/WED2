%! Author = mariuszindel
%! Date = 24.01.21

\section{Web Dev-Ops}


\subsection{Sass (Syntactically Awesome Style Sheets)}
CSS Präprozessoren sind nicht an die Limitationen von CSS gebunden und ermöglichen es uns dadurch, (potentiell) besser wartbaren Code zu schreiben. $\rightarrow$  Weniger Copy \& Paste, Modularisierung, Wiederverwenden von Funktionalitäten.

\subsubsection{SASS vs. SCSS}
2 unterschiedliche Arten von Syntax:
\begin{itemize}
    \item SASS nur mit einrücken
    \item SCSS mit \texttt{;} und \texttt{\{\}}
\end{itemize}

\subsubsection{Variablen}
\begin{lstlisting}
$purpele-navy: #635380; /* definition */
body {
color: $purpele-navy;}  /* Anwendung */
\end{lstlisting}

\subsubsection{Nesting}
\begin{lstlisting}
nav {                   /* SASS */
  ul {
    margin: 0;
      > li {
        display: inline-block; }}}} /* SASS */
nav ul {                /* wird zu CSS */
    margin: 0; }
nav ul > li {           /* wird zu CSS */
display: inline-block; }
\end{lstlisting}

\subsubsection{Mixins}
Ermöglichen das Wiederverwenden von CSS
\texttt{@mixin <name> (<param>*) \{<scss snippet>\}}
\texttt{@include} lädt Mixin (mehrere möglich)
\begin{lstlisting}
@mixin visuallyhidden() {   /* create */
    border: 0;}
.element {                  /* use */
@include visuallyhidden;}
\end{lstlisting}

\subsubsection{Extends / Inheritance}
\begin{itemize}
    \item gleicher Effekt wie mit Mixins
    \item Mixins generiert CSS mit mehr Redundanz
\end{itemize}
\begin{lstlisting}
.icon {
    transition: background-color ease .2s;
.error-icon {
    @extend .icon;
    /* error specific styles... */ }

.icon, .error-icon {
transition: background-color ease .2s; }
.error-icon {
/* error specific styles... */ }
\end{lstlisting}
Wird anstatt \texttt{.icon $\rightarrow$ \%icon} geschrieben, so ist es abstract und es wird kein \texttt{.icon} erzeugt.

\subsubsection{Programmierung}
\begin{lstlisting}
@each $point in $breakpoints {}   /* foreach */
@function properZero($para){      /* function */
    @if( ... ) { @return $para / 5; }}
@each $name, $hex in $colors {    /* Map */
    &-#{$name} {
color: $hex; } }

\end{lstlisting}

\subsubsection{Rechnen}
\begin{tabular}{p{2cm} | l}
    2px + 2px & $\rightarrow$ 4px\\
    10 + px & $\rightarrow$ 10px\\
    1px * 2 & $\rightarrow$ 2px\\
    10px * 10px & $\rightarrow$ «Error»\\
    (2 + px) * 2 & $\rightarrow$ «Error»\\
    2px / 1px & $\rightarrow$ 2\\
\end{tabular}


\subsection{Build Tools}
\textbf{Ziel:} Automatische Optimierung von
\begin{itemize}
    \item Performanz (z.B. first-contentful-paint, time-to-interative) wie sie z.B. Lighthouse
    \item X-Browser Unterstützung
\end{itemize}
\textbf{Features:}
\begin{itemize}
    \item Code Splitting («Tree Shaking»)
    \item X-Browser module-loading optimization
    \item Non-JS Resource loading-optimization
\end{itemize}

\vfill