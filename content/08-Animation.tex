%! Author = mariuszindel
%! Date = 24.01.21

\section{Animation}
% die vermittelte Heuristik anwenden um zu bestimmen ob ein CSS-Property animierbar ist.

\textbf{Animierbar:} length, number, color, etc.\\
\textbf{Nicht Animierbar:} border-style, display, etc.

\subsection{Transition Properties}
% Animationen mit CSS-Transition-Properties (transition-* und transition) definieren, sowie Fehler in vorgegebenen CSS Regel mit Transition-Properties erklären und korrigieren.

% die Unterschiede der wichtigsten Werte des Properties transition-timing-function (linear, ease- in, ease-out, ease-in-out) erklären und einer entsprechenden Visualisierung des Animationsablaufes zuordnen.

\begin{itemize}
    \item \texttt{transition-property:} (Welches Property)
    \begin{itemize}
        \item \texttt{background-color/transform/all/none/}
    \end{itemize}
    \item \texttt{transition-duration:} (Wie lange [s])
    \item \texttt{transition-timing-function:}\\
    (Beschleunigung der Animation)
    \begin{itemize}
        \item \texttt{ease} (Default) startet und endet langsam
        \item \texttt{linear} konstant
        \item \texttt{ease-in} exponentiell
        \item \texttt{ease-out} logarithmisch
        \item \texttt{ease-in-out} wie ease, nur langsamerer start
    \end{itemize}
    \item \texttt{transition-delay:} (Pause vor Start [s])
\end{itemize}
\begin{lstlisting}
.beispiel {
   color: white;
   background-color: blue;
   transition: background-color .3s ease-in-out .2s,
               color 2s ease-in; }
\end{lstlisting}

\subsection{Transform (Bewegen eines Elementes)}
% vorhersagen wie sich CSS transform definitionen (rotate(), rotateX(), rotateY(), translate(), translateX(), translateY(), scale(), scaleX(), scaleY(), skew(), skewX(), skewY(), none) in Kombination mit unterschiedlichen transform-origin Werten (z.B. left, center, right, top) auf die Darstellung eines HTML Elementes auswirken.
\textbf{Values:} rotate(-10deg), rotateX(), rotateY(),  translate()[verschieben], translateX(), translateY(), scale()[vergrössern], scaleX(), scaleY(), skew()[verziehen], skewX(), skewY(), none
\subsubsection{Transform origin}
Drehen um eine bestimmte Achse\\
\textbf{Values:} Percentage, length, left, center, right, top, bottom $\rightarrow$ zuerst X-Achse dann Y-Achse \\
\texttt{transform-origin: 100\% 100\%} $\rightarrow$ unten rechts\\
\texttt{transform-origin: 0\% 0\%} $\rightarrow$ oben links

\begin{lstlisting}
img {
  transition: transform 1s; }

img:hover, img:focus {
  transform: rotate(10deg) translate(50px) scale(1.2);
  transform-origin: 100% 100%; }
\end{lstlisting}

\subsection{Keyframe Animation}
% vorhersagen wie sich eine einfache, mittels einer @keyframe Regel beschriebenen Animation abläuft.
\begin{lstlisting}
#magic {
    animation: rainbow 5s linear infinite; }
    
@keyframes rainbow {
    0% { background-color: red; }
    20% { background-color: orange; }
    40% { background-color: yellow; } ... }
\end{lstlisting}

\subsection{@property Rule}
\begin{lstlisting}
.cprop-anim-gradient-box {
 --color-stop: deeppink; 
 background: linear-gradient(white, var(--color-stop))
 transition: --color-stop 1.5s; }

.cprop-anim-gradient-box:hover {
 --color-stop: deepskyblue; }

@property --color-stop { 
 syntax: '<color>';
 inherits: false; initial-value: transparent; }
\end{lstlisting}

% erklären warum es nötig sein kann ein mit einer @property Definition einen Wertebereich für CSS- Custom Property zu definieren. -> color gradient does not work

% erklären warum die Animation des opacity-Properties gut geeignet ist für Fade-In und Fade-Out Animationen, aber zusätzliche Vorkehrungen getroffen werden müssen um die Accessibility sicher zu stellen. -> screenreader

% erklären warum CSS-Animationen JS-basierten Animationen vorzuziehen sind -> performance

\vfill