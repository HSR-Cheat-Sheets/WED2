%! Author = mariuszindel
%! Date = 24.01.21

\section{Security}


\subsection{XSS}
% Stored XSS
In einem Formular von einem Benutzer eingegebene Daten werden ohne ''Escaping'' an andere Nutzer ausgeliefert.
\textbf{Gegenmassnahmen:}
\begin{itemize}
    \item Encoding bei der Darstellung von Input
    \item Handlebars: {{{unescapedOutput}}}
    \item XSS Libary: \texttt{username: xss(user.username)}
    \item Content Security Policy im Header setzen
    \item Cookies: HTTPOnly Flag wenn möglich setzen
\end{itemize}


\subsection{Code Injection}
% Code-Injection / Remote Code Execution
Angreifer den Server dazu bringen, eingeschleusten Code auszuführen
\textbf{Gegenmassnahmen:}
\begin{itemize}
    \item NICHT eval() (oder setTimeOut(), setInterval()) nutzen sondern parseInt(), JSON.parse()
    \item Im Minimum Input säubern / mittels Regex auf korrektes Format testen
    \item Globale Scopes und Variablen reduzieren.
    \item Rechenintensive Tasks mit childprocess.spawn auslagern
    \item Node NICHT als root Prozess starten
\end{itemize}


\subsection{Broken Authentication}
% Cross-Site Request Forgery
Website mit benutzer-spezifischen Informationen sollten nur für authentisierte Nutzer zugreifbar\\
Passwort oder Token wird nicht verschlüsselt übertragen.
\textbf{Gegenmassnahmen:}
\begin{itemize}
    \item keine geheimen Infos in query-parametern
    \item https (TLS) nutzen
    \item Authentication-Service nutzen (PW-based Auth nicht selber implementieren)
\end{itemize}
Angreifer bringt einen Nutzer mit noch gültiger Session dazu einem gefälschten Formular zu submitten. \textbf{Gegenmassnahmen:}
\begin{itemize}
    \item Formulare bei Ausliefern mit CSRF-Token
    \item Express: Nutzung der csurf-Libarary
\end{itemize}


\subsection{Broken Access Control}

\subsubsection{Insecure Direct Object References}
% Insecure Direct Object References
Bei Anzeige alle Seiten mit benutzerspezifischen Daten sicher stellen,
dass der aktuell eingeloggte Nutzer berechtigt ist $\rightarrow$ Öffentlichen vom geschützten Bereich mit Middleware trennen.

\subsubsection{Replay Attacken}
% Replay Attacken
Bei Spiel sendet der User die Lösung an den Server und dieser überprüft diese.
Problem: Lösung kann mehrmals gesendet werden $\rightarrow$ mehrere Punkte!
Lösung: CSRF Token mit einmaliger Gültigkeit / Gelösste Aufgaben werden gespeichert und nur einmal gezählt

% beurteilen und begründen ob es Sinn macht für eine Anwendung einen passwortbasierten «Authentication Service» selber zu betreiben.

% wichtige generelle Sicherheitsmassnahmen für Node/Express Anwendungen aufzählen, erklären und deren Fehlen im Code einfacher Express-Anwendungen identifizieren und beheben.
\vfill