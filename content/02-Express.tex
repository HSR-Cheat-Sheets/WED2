%! Author = mariuszindel
%! Date = 24.01.21

\section{Express}

\subsection{Middleware}
Express nutzt Middleware für die Request Bearbeitung. Middleware ist ein Stack von Anweisungen, welche für ein Request ausgeführt wird. Mit \texttt{app.use()} wird neue Middleware hinzugefügt.
\begin{lstlisting}
import http from "http";
import express from "express";
const app = express();
const server = http.createServer(app);
const hostname = '127.0.0.1'; const port = '3000'; server.listen(port, hostname, () => { console.log('Running'); });
\end{lstlisting}

\subsubsection{Routing}
Middleware befindet sich auf dem Express Objekt
\begin{lstlisting}
import express from "express";
const router = express.Router;
\end{lstlisting}
router.METHOD(path, [callback, \dots ] callback)  (METHOD : get, put, post, delete)
\begin{lstlisting}
router.get('/', function(req, res) {
res.send('hello world'); });
\end{lstlisting}
Path zu verschiedenen Methoden gruppieren
\begin{lstlisting}
app.route('/book') // oder router.route(..)
.get(function(req, res) {res.send('Get a book');})
.post(function(req, res) {res.send('Add book');})
\end{lstlisting}

\subsubsection{Body-Parser}
\begin{lstlisting}
import bodyParser from "body-parser";
app.use(bodyParser.json());
\end{lstlisting}

\subsubsection{Static}
Statische Files ausliefern, mehrere möglich
\begin{lstlisting}
app.use(express.static(__dirname + '/public'))
\end{lstlisting}

\subsubsection{Eigene Middleware}
3 Parameter: request, response, next
\begin{lstlisting}
function myDummyLogger( options ){
  options = options ? options : {};
  return function myInnerLogger(req, res, next) {
    console.log(req.method +":"+ req.url);
    next(); } }
app.use(myDummyLogger()); //Anwendung
\end{lstlisting}

\subsubsection{Error Middleware}
\begin{itemize}
    \item Bearbeitet Errors, welche von den Middlewares generiert wurden
    \item Wird aufgerufen, falls Error-Objekt dem Next-Callback übergeben wird $\rightarrow$ keine normalen Middlewares mehr aufgerufen
\end{itemize}
\begin{lstlisting}
app.use(function(err, req, res, next) {
  console.error(err.stack);
  res.status(500).send('Something broke!');});
\end{lstlisting}


\subsection{Model (Daten \& Aufbereitung)}
Die Daten sollten in einem Module (Array, JSON, DB) verwaltet und abgespeichert werden.

\subsubsection{nedb}
\begin{lstlisting}
import Datastore from "nedb";
const db = new Datastore({ filename: './order.db', autoload: true });
// insert
db.insert(order, function(err, newDoc){
if (callback){ callback(err, newDoc); } });
// search: findOne oder findAll
db.findOne({ _id: id }, function (err, doc) { callback( err, doc); });
// update
db.update({_id: id}, {$set: {"state": "DELETED"}}, {}, function (err, doc) {
   publicGet(id,callback); });
\end{lstlisting}


\subsection{View (Darstellung der Daten)}
Express bietet eine render Methode an: app.render(view, [locals], callback). Konfiguration der Render-Engine:
\begin{lstlisting}
app.set('view engine', 'hbs'); //Engine
app.set('views', __dirname + '/views'); //Path
\end{lstlisting}
\subsubsection{Handlebars}
\begin{lstlisting}
hbs.registerHelper('if_eq', function(a,b,opts){ if(a===b){ return opts.fn(this); } else{return opts.inverse(this);}
})
// Verwendung im HTML
{{#if_eq state "OK"}}
{{ else }}
{{/ if_eq }}
\end{lstlisting}



\subsection{Session \& Cookie}
Beim ersten Connect vom Client wird eine Session-Id erstellt und als Cookie zum Client geschickt. Session-Data wird auf dem Server abgespeichert $\rightarrow$ wiederspricht REST
\begin{lstlisting}
app.use(require("cookie-parser")());
// Session benoetigt Cookies
app.use(session({ secret: '1234567', resave: false, saveUninitialized: true}));
\end{lstlisting}



\subsection{Rest \& AJAX}
\subsubsection{Token}
Stateless Server: Bei jeder Anfrage muss ein Token mitgeben werden.
\textbf{Vorteil:} Jede Anfrage kann zu einem beliebigen Server gesendet werden solange die Applikationen der Signatur vertraut. \textbf{Nachteile:} Was passiert wenn das Token / Gerät geklaut wird?


\subsubsection{fetch}
\begin{lstlisting}
fetch('/login', { method: 'POST', headers: {
 'Content-Type': 'application/json'},
 body: JSON.stringify({email: "admin@admin.ch",
pwd:"123456"})}).then(function(res){console.log(res);
\end{lstlisting}


\subsection{Websockets}
Klassische Model vom Request-Response hat 2 Probleme:
Server kann keine Nachricht an den Client schicken /
Jede Anfrage öffnete eine neue Verbindung $\rightarrow$ erschwert es real-time Apps z.B. Games\\
\textbf{Lösung:} WebSockets ermöglicht ''bi-directional'', ''always-on'' Kommunikation

\vfill
$ $ 
\columnbreak