%! Author = mariuszindel
%! Date = 24.01.21

\section{Express}

\subsection{Middleware}
Express nutzt Middleware für die Request Bearbeitung. Middleware ist ein Stack von Anweisungen, welche für ein Request ausgeführt wird. Mit \texttt{app.use()} wird neue Middleware hinzugefügt.

\subsubsection{index.js}
Im index.js wird Middleware eingebunden, Routen zugewiesen, Server gestartet
\begin{lstlisting}
import express from 'express';
import bodyParser from 'body-parser';
import hbs from 'express-hbs';
import path from 'path';
import session from 'express-session';
import {indexRoutes} from './routes/indexRoutes.js';

const app = express();
app.engine('hbs', hbs.express4());
app.set('view engine', 'hbs');
app.set('views', path.resolve('views'));

const router = express.Router();
app.use(express.static(path.resolve('public')));
app.use(session({secret: 'abc', resave: false, saveUninitialized: true}));
app.use(bodyParser.urlencoded({extended: false}));
app.use(indexRoutes);

const hostname = '127.0.0.1'; const port = 3001;
app.listen(); });
\end{lstlisting}

\subsubsection{Routing}
Routen ''zeigen'' auf Funktion im Controller
% \begin{lstlisting}
% import express from "express";
% const router = express.Router;
% \end{lstlisting}
\begin{lstlisting}
import express from 'express';
const router = express.Router();
import {indexController} from '../controller/indexController.js';
router.get("/", indexController.index.bind(indexController));
export const indexRoutes = router;
\end{lstlisting}
router.METHOD(path, [callback, \dots ] callback)  (METHOD : get, put, post, delete)
\begin{lstlisting}
router.get('/', function(req, res) {res.send('hello');
\end{lstlisting}
Path zu verschiedenen Methoden gruppieren
% \begin{lstlisting}
% app.route('/book') // oder router.route(..)
% .get(function(req, res) {res.send('Get a book');})
% .post(function(req, res) {res.send('Add book');})
% \end{lstlisting}

% \subsubsection{Body-Parser}
% \begin{lstlisting}
% import bodyParser from "body-parser";
% app.use(bodyParser.json());
% \end{lstlisting}

\subsubsection{Static}
Statische Files ausliefern, mehrere möglich
\begin{lstlisting}
app.use(express.static(__dirname + '/public'))
\end{lstlisting}

% \subsubsection{Eigene Middleware}
% 3 Parameter: request, response, next
% \begin{lstlisting}
% function myDummyLogger( options ){
%   options = options ? options : {};
%   return function myInnerLogger(req, res, next) {
%     console.log(req.method +":"+ req.url);
%     next(); } }
% app.use(myDummyLogger()); //Anwendung
% \end{lstlisting}

% \subsubsection{Error Middleware}
% \begin{itemize}
%     \item Bearbeitet Errors, welche von den Middlewares generiert wurden
%     \item Wird aufgerufen, falls Error-Objekt dem Next-Callback übergeben wird $\rightarrow$ keine normalen Middlewares mehr aufgerufen
% \end{itemize}
% \begin{lstlisting}
% app.use(function(err, req, res, next) {
%   console.error(err.stack);
%   res.status(500).send('Something broke!');});
% \end{lstlisting}


\subsection{Model (Daten \& Aufbereitung)}
Die Daten sollten in einem Module (Array, JSON, DB) verwaltet und abgespeichert werden.

\subsubsection{nedb}
\begin{lstlisting}
import Datastore from "nedb";
const db = new Datastore({ filename: './order.db', autoload: true });
// insert
db.insert(order, function(err, newDoc){
  if (callback){ callback(err, newDoc); } });
// search: findOne oder findAll
  db.findOne({ _id: id }, 
  function (err, doc) { callback( err, doc); });
// update
db.update({_id: id}, {$set: {"state": "DELETED"}}, 
  {}, function (err, doc) {
    publicGet(id,callback); });
\end{lstlisting}


\subsection{View (Darstellung der Daten)}
Express bietet eine render Methode an: app.render(view, [locals], callback). Konfiguration der Render-Engine:
\begin{lstlisting}
app.set('view engine', 'hbs'); //Engine
app.set('views', __dirname + '/views'); //Path
\end{lstlisting}
\subsubsection{Handlebars (hbs)}
% \begin{lstlisting}
% hbs.registerHelper('if_eq', function(a,b,opts){ if(a===b){ return opts.fn(this); } else{return opts.inverse(this);}
% })
% // Verwendung im HTML
% {{#if_eq state "OK"}}
% {{ else }}
% {{/ if_eq }}
% \end{lstlisting}
\begin{lstlisting}
<html>
<img src='/images/pizza.jpg'>
{{#if isLoggedIn}}
    <form action='/orders' method='get'>
    <input type='submit' value='Order a Pizza'>
    </form>
{{else}}
    <form action='/login' method='post'>
      <input type="email" name="email" required>
      <input type="submit" value="Login">
    </form>{{/if}}</html>
\end{lstlisting}
\begin{lstlisting}
<table>
    <thead> <tr> <th>Bezeichnung</th> </tr> </thead>
    <tbody> <tr> <td>{{ type }}</td> </tr> </tbody>
</table>
\end{lstlisting}


\subsection{Controller}
Controller ist für die Logik da, wird beim aufrufen einer Route ausgeführt. Holt sich Daten im Model
\begin{lstlisting}
import {orderStore} from '../services/orderStore.js'
export class OrdersController {
    async showOrder(req, res) {
        res.render("showorder", 
        await orderStore.get(req.params.id)); };}
export const ordersController = new OrdersController()
\end{lstlisting}

\subsection{Session \& Cookie}
Beim ersten Connect vom Client wird eine Session-Id erstellt und als Cookie zum Client geschickt. Session-Data wird auf dem Server abgespeichert $\rightarrow$ wiederspricht REST
\begin{lstlisting}
app.use(require("cookie-parser")());
// Session benoetigt Cookies
app.use(session({ secret: '1234567', resave: false, saveUninitialized: true}));
\end{lstlisting}



\subsection{Rest \& AJAX}
\subsubsection{Token}
Stateless Server: Bei jeder Anfrage muss ein Token mitgeben werden.
\textbf{Vorteil:} Jede Anfrage kann zu einem beliebigen Server gesendet werden solange die Applikationen der Signatur vertraut. \textbf{Nachteile:} Was passiert wenn das Token / Gerät geklaut wird?


\subsubsection{fetch}
\begin{lstlisting}
fetch('/login', { method: 'POST', headers: {
 'Content-Type': 'application/json'},
 body: JSON.stringify({email: "admin@admin.ch",
pwd:"123456"})}).then(function(res){console.log(res);
\end{lstlisting}


\subsection{Websockets}
Klassische Model vom Request-Response hat 2 Probleme:
Server kann keine Nachricht an den Client schicken /
Jede Anfrage öffnete eine neue Verbindung $\rightarrow$ erschwert es real-time Apps z.B. Games\\
\textbf{Lösung:} WebSockets ermöglicht ''bi-directional'', ''always-on'' Kommunikation

\vfill
$ $ 
\columnbreak